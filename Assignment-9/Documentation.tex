\documentclass[11pt]{article}
\usepackage{natbib,mybigpackage}
\usepackage{algorithm}
%\usepackage{program}
%\usepackage{algpseudocode}
\usepackage{algorithmic}
\usepackage{listings}
\def\xbf{\mathbf{x}}
\def\zbf{\mathbf{z}}
\def\xibf{\mathbf{\xi}}
\title{Documentation: Assignment 9}
\author{Abhinav Gupta -- 150123001}
\begin{document}
\titlepage
\newpage
\begin{enumerate}
\item[Q 1] Multivariate normal (Make empirical contour plots based on above generated samples.)
\end{enumerate}
\noindent{Code: R}
\begin{lstlisting}
library(MASS)	
x2<-c()
count<-1000
a<- c(-0.25,0,0.25)
for(i in a){
	sigma<-matrix(c(1,2*i,2*i,4),nrow=2)
	mean<-matrix(c(5,8),nrow=2)
	z1<-rnorm(count)
	z2<-rnorm(count)
	x1<-mean[1,1]+(sigma[1,1]^.5)*z1
	x2<-mean[2,1]+(i*sigma[2,2]^.5)*z1+(((1-(i^2))^.5)*(sigma[2,2]^.5))*z2
	mean_final<-matrix(c(mean(x1),mean(x2)),nrow=2)
	sigma_final<-matrix(c(var(x1),cov(x1,x2),cov(x1,x2),var(x2)),nrow=2)
	cat("\nMean Matrix is: \n")
	cat(mean_final[1,],"\n")
	cat(mean_final[2,],"\n")
	cat("\nSigma Matrix is: \n")
	cat(sigma_final[1,],"\n")
	cat(sigma_final[2,],"\n")	
	z<-kde2d(x1,x2)
	png(paste("question1-a: ",toString(i),".png",sep=""))
	filled.contour(z)	
}

\end{lstlisting}
\noindent{\textbf{Output}:}\

Mean Matrix is: \

4.97803 \

7.848047 \


Sigma Matrix is: \

0.9939227 -0.516966\
 
-0.516966 4.147666 \


Mean Matrix is: \

4.978293 \

7.92619 \


Sigma Matrix is: \

0.9829786 -0.1517543 \

-0.1517543 4.142172\
 

Mean Matrix is: \

5.004119 \

8.065482 \


Sigma Matrix is: \

1.010186 0.4767248 \

0.4767248 3.81206 \


\noindent{\textbf{Observation}:}
\textbf{Graph: a= -0.25 : }\
\begin{figure}[H]
  \centering
 \subfloat[X1]{\includegraphics[width=.6\textwidth]{{question1-a=-0.25}.png}}\\
\end{figure}
\textbf{Graph: a= 0 : }\
\begin{figure}[H]
  \centering
 \subfloat[X1]{\includegraphics[width=.6\textwidth]{{question1-a=0}.png}}\\
\end{figure}
\textbf{Graph: a= 0.25 : }\
\begin{figure}[H]
  \centering
 \subfloat[X1]{\includegraphics[width=.6\textwidth]{{question1-a=0.25}.png}}\\
\end{figure}
%----------------------------------------------------------------------------------------------------------
\newpage
\begin{enumerate}
\item[Q 2] Also, plot the actual and empirical marginal cdfs of X1 and X2.
\end{enumerate}
\noindent{Code: R}
\begin{lstlisting}
library(MASS)
x2<-c()
count<-1000
a<- c(-0.25,0,0.25)
for(i in a){
  sigma<-matrix(c(1,2*i,2*i,4),nrow=2)
  mean<-matrix(c(5,8),nrow=2)
  z1<-rnorm(count)
  z2<-rnorm(count)
  x1<-mean[1,1]+(sigma[1,1]^.5)*z1
  x2<-mean[2,1]+(i*sigma[2,2]^.5)*z1+(((1-(i^2))^.5)*(sigma[2,2]^.5))*z2
  mean_final<-matrix(c(mean(x1),mean(x2)),nrow=2)
  sigma_final<-matrix(c(var(x1),cov(x1,x2),cov(x1,x2),var(x2)),nrow=2)
  cat("\nMean Matrix is: \n")
  cat(mean_final[1,],"\n")
  cat(mean_final[2,],"\n")
  cat("\nSigma Matrix is: \n")
  cat(sigma_final[1,],"\n")
  cat(sigma_final[2,],"\n")
  absolute<-mvrnorm(count,mean,sigma)		
  string<-paste("question2- X1 - a: ",toString(i),".png",sep="")
  png(string)
  plot(ecdf(absolute[,1]),col="green",main="Green is actual and Red is emperical ")
  par(new=TRUE)
  plot(ecdf(x1),col="red")
  string<-paste("question2- X2 - a: ",toString(i),".png",sep="")
  png(string)
  plot(ecdf(absolute[,2]),col="green",main="Green is actual and Red is emperical ")
  par(new=TRUE)
  plot(ecdf(x2),col="red")
  
}

\end{lstlisting}
\noindent{\textbf{Output}:}\


Mean Matrix is: \

4.959824 \

7.949048 
\

Sigma Matrix is: \

0.9955896 -0.5380006 \

-0.5380006 4.272224 \


Mean Matrix is: \

5.04837 \

8.169231 \


Sigma Matrix is: \

0.9958008 0.1126358 \

0.1126358 4.13406 \


Mean Matrix is: \

4.993082\
 
8.052734 \


Sigma Matrix is: \

1.044344 0.4855019\
 
0.4855019 4.331652 \
\newpage

\noindent{\textbf{Observation}:}
\textbf{Graph: a= -0.25 : }\
\begin{figure}[H]
  \centering
 \subfloat[X1]{\includegraphics[width=.6\textwidth]{{question2- X1 - a=0.25}.png}}\\
\end{figure}
\textbf{Graph: a= -0.25: }\
\begin{figure}[H]
  \centering
 \subfloat[X2]{\includegraphics[width=.6\textwidth]{{question2- X2 - a=-0.25}.png}}\\
\end{figure}
\textbf{Graph: a= 0 : }\
\begin{figure}[H]
  \centering
 \subfloat[X1]{\includegraphics[width=.6\textwidth]{{question2- X1 - a=0}.png}}\\
\end{figure}
\textbf{Graph: a= 0: }\
\begin{figure}[H]
  \centering
 \subfloat[X2]{\includegraphics[width=.6\textwidth]{{question2- X2 - a=0}.png}}\\
\end{figure}
\textbf{Graph: a= 0.25 : }\
\begin{figure}[H]
  \centering
 \subfloat[X1]{\includegraphics[width=.6\textwidth]{{question2- X1 - a=0.25}.png}}\\
\end{figure}
\textbf{Graph: a= 0.25: }\
\begin{figure}[H]
  \centering
 \subfloat[X2]{\includegraphics[width=.6\textwidth]{{question2- X2 - a=0.25}.png}}\\
\end{figure}
%---------------------------------------------------------------------------
\newpage
\begin{enumerate}
\item[Q 3] generating a bivariate normal with the help of conditional distributions.
\end{enumerate}
\noindent{Code: R}
\begin{lstlisting}
x2<-c()
count<-1000
a<- c(-0.25,0,0.25)
for(i in a){
	sigma<-matrix(c(1,2*i,2*i,4),nrow=2)
	mean<-matrix(c(5,8),nrow=2)
	z1<-rnorm(count)
	z2<-rnorm(count)
	x2<-mean[2,1]+(sigma[2,2]^.5)*z1
	mean_intermediate<-mean[1,1]+i*((sigma[1,1]^.5)/(sigma[2,2]^.5))*(x2-mean[2,1])
	sd<-sigma[1,1]*(1-i^2)^.5
	x1<-mean_intermediate+(sd^.5)*z2
	mean_final<-matrix(c(mean(x1),mean(x2)),nrow=2)
	sigma_final<-matrix(c(var(x1),cov(x1,x2),cov(x1,x2),var(x2)),nrow=2)
	cat("\nMean Matrix is: \n")
	cat(mean_final[1,],"\n")
	cat(mean_final[2,],"\n")
	cat("\nSigma Matrix is: \n")
	cat(sigma_final[1,],"\n")
	cat(sigma_final[2,],"\n")
}

\end{lstlisting}
\noindent{\textbf{Output}:}\


Mean Matrix is: \

4.989924 \

7.919084 \


Sigma Matrix is: \

1.008933 -0.4903051 \

-0.4903051 3.984307 \


Mean Matrix is: \

4.983557 \

8.013008 \


Sigma Matrix is: \

0.9888262 -0.04890436 \

-0.04890436 3.851085 \


Mean Matrix is: \

4.980537 \

8.022619 \


Sigma Matrix is: \

0.9542868 0.4811441 \

0.4811441 3.808051\

\end{document}