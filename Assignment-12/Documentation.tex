\documentclass[11pt]{article}
\usepackage{natbib,mybigpackage}
\usepackage{algorithm}
%\usepackage{program}
%\usepackage{algpseudocode}
\usepackage{algorithmic}
\usepackage{listings}
\def\xbf{\mathbf{x}}
\def\zbf{\mathbf{z}}
\def\xibf{\mathbf{\xi}}
\usepackage{lmodern}% http://ctan.org/pkg/lm
\title{Documentation: Assignment 12}
\author{Abhinav Gupta -- 150123001}
\begin{document}
\titlepage
\newpage
\begin{enumerate}
\item[Q 1]  Generate the first 25 values of the Van der Corput sequence \(x_1,x_2,...,x_25\) using the
radical inverse function \(x_i:=\phi_2(i)\) and list them in your report. Next, generate the
first 1000 values of this sequence and plot the overlapping pairs \((x_i,x_{i+1})\) as a two
dimensional plot. What do you observe ? Now, generate first 100 and 100000 values
of this sequence and plot the sampled distributions for both the cases. Compare these
plots with the sampled distributions of 100 and 100000 values generated by an LCG,
by plotting the sampled distributions in two graphs side by side for both the cases.
Specify the LCG that you have used.
\end{enumerate}
\noindent{Code : R}
\begin{lstlisting}
lcg<-function(a,x,m,count){
	u<-c()
	for(i in 1:count){
		u[i]=x/m
		x=(x*a)
		x=x%%m
	}
	return(u)
}
vanDerCorput<-function(n,base){
	sum=0
	temp=1/base
	while(n!=0){
		sum=sum+(n%%base)*temp
		n=n%/%base
		temp=temp/base
	}
	return(sum)
}
count<-25
cat("Number of sequence elements= ",count,"\n")
sample<-c()
for(i in 1:count){
	sample[i]<-vanDerCorput(i,2)
	cat(sample[i],"\n")
}
count<-1000
sample<-c()
sample1<-c()
png(paste("question1_",count,".png",sep=""))
for(i in 1:count){
	sample[i]<-vanDerCorput(i,2)
	if(i>1){
		sample1[i-1]<-sample[i]
	}
}
sample1[count]<-vanDerCorput(count+1,2)
plot(sample,sample1)
counts<-c(100,100000)
x<-23
a<-16807
m<-2^31-1
for(count in counts){
	sample<-c()
	for(i in 1:count){
		sample[i]<-vanDerCorput(i,2)
	}
	sample1<-lcg(a,x,m,count)
	png(paste("question1_",toString(count),".png",sep=""))
	par(mfrow=c(1,2))
	hist(sample,main="Van Der Corput",breaks=20,col="blue",ylab="Frequency",)
	hist(sample1,main="LCG",breaks=20,col="red",ylab="Frequency",)
}

\end{lstlisting}
\newpage
\noindent{\textbf{Output}:}\\
\textbf{First 25 values of Van Der Corput sequence: }\\
0.5\\ 
0.25\\ 
0.75\\ 
0.125\\ 
0.625\\ 
0.375\\ 
0.875\\ 
0.0625\\ 
0.5625\\ 
0.3125\\ 
0.8125\\ 
0.1875\\ 
0.6875\\ 
0.4375\\ 
0.9375\\ 
0.03125\\ 
0.53125\\ 
0.28125\\ 
0.78125\\ 
0.15625\\ 
0.65625\\ 
0.40625\\ 
0.90625\\ 
0.09375\\ 
0.59375\\ 
\newpage
\begin{figure}[H]
\textbf{Graph of overlapping pairs \((x_i,x_{i+1})\) for 1000 values: }\\
\centering
\subfloat[X1]{\includegraphics[width=.6\textwidth]{{question1_1000}.png}}
\end{figure}
\begin{description}
\textbf{Graph: }\\
\textbf{LCG used is\ \((ax)mod(m)/m \ where \ a=16807,\ x_0=23\ and\ m=2^31-1 \)}
\begin{figure}[H]
\item Sampled Distributions for 100 values\\
\centering
\subfloat[X1]{\includegraphics[width=.6\textwidth]{{question1_100}.png}}
\end{figure}
\begin{figure}[H]
\item Sampled Distributions for 100000 values\\
\centering
\subfloat[X1]{\includegraphics[width=.6\textwidth]{{question1_1e+05}.png}}
\end{figure}

\end{description}
%----------------------------------------------------------------------------------------------------------------
\newpage
\begin{enumerate}
\item[Q 2]  Generate the Halton sequence \(x_i=(\phi_2(i),\phi_3(i))\) (as points in \(R_2\) ) and plot the first
100 and 100000 values. What are your observations ? Recall that the radical inverse function is defined by \[\phi_b(i)=
\sum_{k=0}^jd_kb^{-k-1} where \ i=\sum_{k=0}^jd_kb^{k}\]
\end{enumerate}
\noindent{Code : R}
\begin{lstlisting}
vanDerCorput<-function(n,base){
	sum=0
	temp=1/base
	while(n!=0){
		sum=sum+(n%%base)*temp
		n=n%/%base
		temp=temp/base
	}
	return(sum)
}

counts<-c(100,100000)
for(count in counts){
	sample2<-c()
	sample3<-c()
	for(i in 1:count){
		sample2[i]<-vanDerCorput(i,2)
		sample3[i]<-vanDerCorput(i,3)
	}
	png(paste("question2_",count,".png",sep=""))
	plot(sample2,sample3,xlab="Base 2",ylab="Base 3")
}

\end{lstlisting}
\noindent{\textbf{Observation}:}\\
We are getting all points in range [0,1] and more closer to uniform. Plus as we are generating in higher dimension 
i.e. 2, so we will get net variance/error less than that of numeric method.
\newpage 
\begin{description}
\textbf{Graph: }\
\begin{figure}[H]
\item For 100000 values\\
\centering
\subfloat[X1]{\includegraphics[width=.6\textwidth]{{question2_1e+05}.png}}
\end{figure}
\begin{figure}[H]
\item For 100 values\\
\centering
\subfloat[X1]{\includegraphics[width=.6\textwidth]{{question2_100}.png}}
\end{figure}
\end{description}
\end{document}
